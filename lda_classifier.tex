\section{LDA as classifier}

Classification is one of the fundamental tasks in machine learning and statistical analysis. At its core, a classifier is a function that takes input features and assigns them to predefined categories or classes. The goal is to develop models that can make accurate predictions on new, unseen data by learning patterns from training examples. While there are many approaches to classification, from simple rule-based systems to complex neural networks, we're particularly interested in understanding how well a classifier performs and what makes a classifier optimal. This leads us to a formal framework for evaluating classifier performance through the concept of prediction error.

Our approach to developing an effective classifier will follow a principled path: we'll first establish the theoretical foundations using probability theory to understand what makes a classifier optimal. This will lead us to the concept of posterior probabilities and their role in classification decisions. Then, we'll bridge the gap between theory and practice by showing how these theoretical insights can be implemented using training data and linear algebra techniques, ultimately arriving at Linear Discriminant Analysis (LDA) as a practical and powerful classification method.
