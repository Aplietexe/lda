\section{LDA como clasificador}

La clasificación constituye una de las tareas fundamentales del aprendizaje automático y el análisis estadístico. En esencia, un clasificador es una función que asigna características de entrada a categorías o clases predefinidas. El objetivo es desarrollar modelos capaces de realizar predicciones precisas sobre datos nuevos y no vistos, mediante el aprendizaje de patrones en ejemplos de entrenamiento. Si bien existen diversos enfoques para la clasificación, desde sistemas basados en reglas simples hasta redes neuronales complejas, nos interesa particularmente comprender la eficacia de un clasificador y qué lo hace óptimo. Esto nos conduce a un marco formal para evaluar el rendimiento del clasificador a través del concepto de error de predicción esperado.

Primero estableceremos las bases teóricas mediante Teoría de Probabilidad para entender qué hace que un clasificador sea óptimo. Esto nos llevará al concepto de probabilidades a posteriori y su papel en las decisiones de clasificación. Posteriormente, pasaremos a la práctica mostrando cómo estos conocimientos teóricos pueden implementarse utilizando datos de entrenamiento y técnicas de álgebra lineal, llegando finalmente al Análisis Discriminante Lineal (LDA) como método de clasificación práctico.
